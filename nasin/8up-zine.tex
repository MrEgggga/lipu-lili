\documentclass[statementpaper,oneside,article,14pt]{memoir}
\usepackage{geometry}
\usepackage{libertine}
\usepackage{kantlipsum}

\usepackage{enumitem}
\usepackage{fontspec}

\setmainfont{Futura}
\setmonofont{Fira Code}
\newfontfamily\sitpon{nasin-nanpa}
\newfontfamily\sitponalt{linja sike 5}

\usepackage{titlesec}

% Disable chapter/section numbering.
\setsecnumdepth{none}
\maxsecnumdepth{none}

% Optional background
% http://tex.stackexchange.com/a/276280
\usepackage{transparent}
\usepackage{eso-pic}
\newcommand{\BackgroundPic}[1]{%
\put(0,0){%
\parbox[b][\paperheight]{\paperwidth}{%
\vfill
\centering
{\transparent{0.4} \includegraphics[width=\paperwidth,height=\paperheight,%
keepaspectratio]{#1}}%
\vfill
}}}

\setlength{\parindent}{1em}

\begin{document}

\titlespacing\subsection{0pt}{0pt}{0pt}

\newcommand{\nimi}[3]{\item[{\sitpon#1} #2] #3}

% Edit inside the { brackets } to change these.

\title{{\sitpon{󱤪󱤨󱦓󱤿󱥬}} \\ lipu lili pi nasin toki}
\author{{\sitpon󱤑󱦐󱥚󱦜󱤻󱦜󱦑} \\ jan Semu}
\date{13 May 2022}

\begingroup
\let\cleardoublepage\clearpage

% \AddToShipoutPicture*{\BackgroundPic{samplecover}}

\begin{titlingpage}
\maketitle

% Could add a small author's note, etc. here if you like.

\end{titlingpage}

\endgroup

% As the zine is so short, you probably won't need page numbers; however, if you
% want them, comment out the next line with a %.
\pagestyle{empty}


%% CONTENT GOES BELOW

\subsection{{\sitpon󱤧} basic sentences}

\noindent A basic sentence in toki pona consists of a subject and predicate. These are separated by li ({\sitpon󱤧}).
If the subject is mi or sina ({\sitpon{󱤴󱥞}}) alone, li must be omitted.

\begin{quote}
  ona li pona. (They are good.)

  mi moku. (I eat.)
\end{quote}

\subsection{{\sitpon󱥄} addressing}

\noindent To make a sentence imperative, replace li with o ({\sitpon{󱥄}}). To mark it as vocative, you can put a comma after the o.

\begin{quote}
  o moku. (Please eat.)

  jan o, toki! (Hello, person!)
\end{quote}

\subsection{{\sitpon{󱤉}} direct objects}

\noindent To specify a direct object, use e ({\sitpon{󱤉}}) after li.

\begin{quote}
  mi moku e kili. (I eat the fruit.)

  ona li toki e toki pona. (They speak about toki pona.)
\end{quote}

\noindent Almost every content (non-particle) word \\ can be used as a predicate.

\begin{quote}
  mi pona e ijo. (I fix the thing.)

  mi luka e nena. (I apply hand to the button; I press the button.)
\end{quote}

\subsection{{\sitpon󱤊} compound sentences}

\noindent Toki Pona has no way of combining two separate sentences. However, you can use li multiple times 
to specify multiple predicates, or e multiple times to specify multiple objects. To specify 
multiple subjects, use en ({\sitpon󱤊}).

\begin{quote}
  mi en sina li lukin li kute e pipi e jan. (You and I look at and listen to bugs and people.)
\end{quote}


\subsection{{\sitpon󱥍} adjectives}

\noindent Adjectives come after the nouns or verbs they modify. Each is applied left-to-right, i.e.\ A B C is interpreted as (A B) C. To regroup adjectives, use pi ({\sitpon󱥍}). It is generally uncommon to use pi multiple times in the same clause (pi-stacking).

\begin{quote}
  ilo kalama (instrument)

  ilo kalama suli (big instrument)

  ilo pi kalama suli (loud tool)
\end{quote}

Predicate clauses often have adjectives attached---in this case the adjectives function as adverbs.

\begin{quote}
  ona li toki pona. (They speak well. / They speak Toki Pona.)
\end{quote}

\subsection{{\sitpon󱤡} la}

\noindent The particle la ({\sitpon󱤡}) is called the ``context'' particle. 
``A la B'' roughly means ``In the context of A, B.'' this can be used for time, adverbs, etc.

\begin{quote}
  tenpo kama la mi lape. (I sleep in the context of coming time; I will sleep.)

  ken la mi lape. (I might sleep.)

  mi moku e kili la mi lape. (I will sleep if I eat fruit.)
\end{quote}

\subsection{{\sitpon󱤬} prepositions}

\noindent Some words (marked with \textit{prep.} in lipu lili pi nimi ale) in toki pona 
function as prepositions (e.g.\ lon, tawa, kepeken, nanpa).

\begin{quote}
  ona li lon. (They exist.)

  mi lon tomo tawa. (I'm in the car.)

  mi pana e kili tawa sina. (I give you a fruit.)
\end{quote}

If the predicate of a sentence is a preposition, using e makes the sentence transitive (the object is the thing that the preposition applies to).

\begin{quote}
  mi tawa supa. (I move towards the table.)

  mi tawa e supa. (I move the table.)

  mi kepeken ilo. (I use the tool.)

  * mi kepeken e ilo. (I can't think of a good translation for this.)
\end{quote}

\noindent Position words (e.g.\ poka, monsi) can be turned into prepositions by putting lon before them.

\begin{quote}
  mi lon poka pi tomo tawa. (I'm near the car.)
\end{quote}

\subsection{{\sitpon󱥷} preverbs}

\noindent Some words (e.g.\ wile, sona, kama) are \textit{preverbs}, meaning they are put before a verb
to modify it. (These verbs are marked with \textit{pv.} in lipu lili pi nimi ale and lipu Linku.)

\begin{quote}
  mi wile moku. (I want to eat.)

  mi ken pali e lipu. (I can make books.)
\end{quote}

\subsection{{\sitpon󱤑} names}

\noindent All proper nouns in toki pona are marked with capitalization (or cartouches in sitelen pona).
Proper nouns function as adjectives, meaning they must be prefixed with a noun that describes 
the thing they refer to. It is common to revise these names to match toki pona's phonotactics.

\begin{quote}
  jan Semu (person named Semu)

  toki Epelanto (language named Epelanto)
\end{quote}

To convert proper nouns into toki pona, follow these rules:

\begin{enumerate}[itemsep=-5pt]
\item Prioritize phonetics over spelling.
\item Use endonyms instead of exonyms.
\item Use local pronounciations rather than ``official'' ones.
\item Follow toki pona phonotactics (shown on last page).
\end{enumerate}

\noindent See \textit{jan-ne.github.io/tp/tpize} for a full list.

\subsection{{\sitpon󱥙} questions}

\noindent Yes or no questions in toki pona can either be formed by saying a word twice with ala in 
between or by adding ``anu seme'' at the end of the sentence. These questions can be answered 
by repeating the predicate for yes, and repeating the predicate followed by ala for no.

\begin{quote}
  kili li pona ala pona tawa sina? (do you like fruit?)

  kili li pona tawa sina, anu seme? (do you like fruit?)
\end{quote}
Open-ended questions use ``seme'' ({\sitpon󱥙}), a word that functions similar to
English words ``what'' and ``which''.

\subsection{{\sitpon󱤕} phonotactics}

toki pona has the consonants \textit{mnptkswlj} and the vowels \textit{iueoa}.

Syllables in toki pona use \textit{CV(N)}, where \textit{C} is a consonant 
(optional for the first syllable of a word), \textit{V} is a vowel, and \textit{N} is 
an optional \textit{n}.

The syllables \textit{wuwojiti} are disallowed, as well as the same syllables with added 
\textit{n}. No two nasals can appear next to each other (i.e., no \textit{nm} or \textit{nn}).

\end{document}